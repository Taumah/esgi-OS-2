\documentclass[11pt,a4paper,addpoint]{exam}
\usepackage[T1]{fontenc}
\usepackage[utf8]{inputenc}
\usepackage[]{lmodern}
\usepackage[french]{babel}
%\usepackage{mathtools}
\usepackage[margin=2cm]{geometry} %layout
\usepackage{graphicx}
\usepackage{booktabs} % for much better looking tables
% Put the bibliography in the ToC
\usepackage[nottoc,notlof,notlot]{tocbibind}
% Alter the style of the Table of Contents
\usepackage[titles]{tocloft}

\usepackage[pdfauthor={Axel Viala},
  pdftitle={Examen ESGI OS-01},
  pagebackref=true,%
  colorlinks=true,%
  linkcolor=green,%
  %urlcolor=green!70!black,
  pdftex]{hyperref}
\usepackage[ampersand]{easylist}
\renewcommand{\solutiontitle}{}

\nopointsinmargin
\pointformat{}

\author{Axel Viala <axel.viala@darnuria.eu>}
\title{2019-10-17: Auto evaluation connaissances 01 - durée 15 minutes}

\begin{document}
  \maketitle
  \makebox[\textwidth][l]{Nom et Prénom:\hrulefill}
  \makebox[\textwidth][l]{CorrectChoiceeur: Nom et Prénom:\hrulefill}
  \makebox[\textwidth][l]{Classe:\hrulefill}  
  \textbf{Objectifs:} Le but des contrôles de connaissances en début de cours est pour vous de vérifier ou vous
  en êtes par rapport au cours précédent.
  \newline
  Pour moi un moyen de vérifier que la pédagogie est adaptée à la classe.
  \newline
  \textbf{Notation:}: Les points sont indiqué à titre d'information la notation peut changer pour
  des raisons d'harmonisation.
  \newline
  \textbf{Entre-correction:} Vous pourrez faire la correction d'une autre copie avec tout document
  autorisé aura lieu au retour de la pause si vous arrivez à l'heure. La correction
  améliore la note initiale du correcteur.
  \begin{questions}
    \question[0] Avez-vous déjà fait du C: Si oui combien de temps environ.
    \vspace{64pt}

    \question[2] Un descripteur de fichier \emph{file descriptor} est sous unix : (plusieurs choix possibles)
    \begin{checkboxes}
      \choice Un pointeur
      \CorrectChoice Un identifiant abstrait utilisable pour accéder a une ressource (ex-fichier)
      \CorrectChoice Représenté par un simple entier positif 0 inclu.
      \choice Une structure regroupant des informations sur un fichier
    \end{checkboxes}

    \question[3] Le langage C est un langage (plusieurs choix possibles)
      \begin{checkboxes}
        \CorrectChoice Compilé
        \CorrectChoice La gestion de la mémoire (allocations/desallocations) est gérée manuellement
        \choice Exécuté en machine virtuelle comme Python
        \choice À typage dynamique
        \CorrectChoice À typage statique
      \end{checkboxes}

    \question[5] Un appel système est : (plusieurs choix possibles)
    \begin{checkboxes}
      \CorrectChoice Est en général un service proposé par le kernel du système d'exploitation
      \CorrectChoice Son appel entrainera la suspension de mon programme pour passer en mode kernel
      \CorrectChoice \texttt{write} sous linux est un appel système
      \choice \texttt{sizeof} est un appel système
      \CorrectChoice Les appels systèmes sont des fonctions comme les autres fonctions en C
      \CorrectChoice Un appel système nécessite une instruction particulière du processeur
    \end{checkboxes}

    \question[3] En C \texttt{"Hello world"} peut être du type: (plusieurs choix possibles)
    \begin{checkboxes}
      \CorrectChoice \texttt{char *}
      \choice \texttt{int}
      \CorrectChoice \texttt{char[]}
      \CorrectChoice \texttt{const char[]}
    \end{checkboxes}

    \newpage
    \question[1] Le type \texttt{int * a} représente: (un seul choix possible)
    \begin{checkboxes}
      \CorrectChoice Un pointeur sur un entier
      \choice Un entier
      \choice Un pointeur de pointeur
      \choice La réponse D
    \end{checkboxes}


    \question[3] Le programme \texttt{objdump} sert à: (un seul choix possible)
    \begin{checkboxes}
      \CorrectChoice Analyser un programme en langage machine vers de l'assembleur
      \CorrectChoice Lire du code source (en ASM)
      \CorrectChoice Tracer les appels systèmes d'un programme
    \end{checkboxes}

    \question[1] Un programme C est compilé par gcc en: (un seul choix possible)
    \begin{checkboxes}
      \choice Klingon
      \CorrectChoice Langage machine
      \choice Byte code de haut niveau pour machine virtuelle
    \end{checkboxes}

    \question[3] La commande \texttt{man} sert à: (plusieurs choix possibles)
    \begin{checkboxes}
      \CorrectChoice Obtenir diverses documentations notamment celles des appels systèmes
      \CorrectChoice Lire des exemples de code pour utiliser un syscall
      \CorrectChoice Connaitre des options sur une commande de shell
    \end{checkboxes}

    \question[2] Le type \texttt{void} (sans *) dans une signature de fonctions en C peut indiquer: (plusieurs choix possibles)
    \begin{checkboxes}
      \CorrectChoice Que une fonction ne prends pas de paramètres
      \CorrectChoice Que une fonction ne retourne pas de valeur
    \end{checkboxes}

    \question[3] L'appel système \texttt{write} attends trois paramètres citez en au moins deux.
    \ifprintanswers
    \begin{solution}
        \begin{itemize}
            \item Descripteur de fichier sur lequel on va écrire 
            \item Adresse du début du message (bloc de mémoire contingue) à écrire
            \item nombres de bytes à écrire de notre message      
        \end{itemize}
    \end{solution}  
    \else
        \vspace{64pt}
    \fi

    \question[2] Citez au moins deux ressources rendues accessibles par descripteur de fichier par le kernel linux:
    \ifprintanswers
    \begin{solution}
        \begin{itemize}
            \item Fichiers sur le disque
            \item Périphériques
            \item (bonus) Information sur les processus
            \item (bonus) Réseau
            \item (bonus) intercommunication entre processus
            \item (bonus) mémoire
        \end{itemize}
    \end{solution}
    \else
    \vspace{64pt}
    \fi

    \question[5] Un kernel comme linux propose (plusieurs choix possibles):
    \begin{checkboxes}
      \CorrectChoice Des abstractions pour gérer du matériel
      \CorrectChoice De quoi communiquer entre programmes
      \choice De quoi garantir qu'une variable est immutable
      \CorrectChoice Pouvoir envoyer des données sur le réseau
      \CorrectChoice Des mécanismes pour allouer dynamiquement de la mémoire
      \CorrectChoice Gérer que lorsque l'on appuie sur une touche notre programme puissent être notifié
    \end{checkboxes}
  \end{questions}
\end{document}
